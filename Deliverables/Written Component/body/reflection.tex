While it seems as though the LSTM outperforms both models, the decision cannot be made without some level of uncertainty on the forecasts obtained. Prediction intervals can be obtained with ease for the Holt-Winters and ARIMA model, but such intervals are not as easily obtained for the LSTM. To do this, the framework of the LSTM would have to be retrained on bootstrapped samples of the original training data and used to make predictions. This could be repeated 1,000 times to obtain prediction intervals. This process, however, would require a lot of time and is out of the scope of this paper. As the LSTM being compared required over 20,000 iterations for a single prediction, the process of obtaining uncertainty for the predictions would take quite some time.

Another issue that arises is the loss of interpretation of these results. For example, the ARIMA model had a $4^{th}$ order auto-regressive component, which indicates that the water depth at any given day is mostly related to the water depth of the past 4 days. Inference like this cannot be made with the LSTM.

For future research, uncertainty intervals should be obtained to determine if the LSTM is able to obtain higher levels of certainty regarding where a time series will be in the future when compared to other methods. Another step that could be taken for future work is to compare additional methods for time series forecasting. While ARIMA and Holt-Winters are common and effective, there is no shortage of methods that can be used to forecast time series.

Another interesting route that could be explored is to implement a genetic algorithm to randomly mutate the a subset of high-performing models to create new generations. In Figure \ref{fig:Variability}, it can be seen that 3 of the 5 models perform very well. A genetic algorithm would allow for some of the components of each model to be taken and turned into a new generation of models. From there, the next top performers could be selected and used for future generations. This process would work best when all models have differing structures, although that was not performed for this analysis.