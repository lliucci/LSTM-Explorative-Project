Complex networks such as the LSTM require large amounts of training data in order to maximize performance. Ideally, the data would have low missingness, as missing values can impact training effectiveness (this is true for all networks, although the issue becomes more pronounced when data are time-dependent). Additionally, the data should not have any seasonal increase over time. Due to these requirements, the training data were selected to be water depth time series collected within the Everglades National Park (EVER). There are 52 monitoring stations present, all with various levels of missingness. The chosen station, P33, had less than 1\% missingness and had only a minor increasing trend (see Figure \ref{fig:P33}).

\begin{figure}[ht]
    \centering
    \includegraphics[width=0.9\linewidth]{"Figures/P33_Time_Series_Missingness.png"}
    \caption{A line plot of the time series used to train the LSTM in this project. Red vertical bars indicate a date with a missing value.}
    \label{fig:P33}
\end{figure}

Due to the present although low missingness, an interpolation method had to be chosen to impute those missing values. Mean interpolation, linear interpolation, and polynomial interpolation were all considered \citep{lepot2017interpolation}. These methods work well in situations where the gaps to be filled are small and of uniform size, which is not the case for the data chosen. Ultimately, the decision to use Kalman smoothing was made as it functions best in the case of noisy, realistic data with missingness intervals of varying size \citep{kalmanfilter}. The line plot in Figure \ref{fig:P33_Interpolated} displays the imputed values. Note that this plot focuses on 1980 to 2000, despite imputation being performed across the entire time series.

\begin{figure}[ht]
    \centering
    \includegraphics[width=0.9\linewidth]{"Figures/Interpolation_60_20.png"}
    \caption{A line plot of the time series used to train the LSTM in this project. Red lines indicate the imputed values using the Kalman filter.}
    \label{fig:P33_Interpolated}
\end{figure}

\FloatBarrier